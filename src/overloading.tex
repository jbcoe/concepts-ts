
%%
%% Overloading
%%
\setcounter{chapter}{12}
\rSec0[over]{Overloading}

Modify paragraph 1 to allow overloading based on constraints.

\begin{quote}
\pnum
When two or more different declarations are specified for a single name 
in the same scope, that name is said to be overloaded. By extension, two 
declarations in the same scope that declare the same name but with
different types \added{or different associated constraints (\ref{temp.constr.decl})}
are called \defn{overloaded declarations}. Only function and function 
template declarations can be overloaded; variable and type declarations 
cannot be overloaded.
\end{quote}

%%
%% OVerloadable declarations
%%
\rSec1[over.load]{Overloadable declarations}

Update paragraph 3 to mention a function's overloaded constraints. 
Note that the itemized list in the original text is omitted in this
document.

\begin{quote}
\setcounter{Paras}{2}
\pnum
\enternote
As specified in \ref{dcl.fct}, function declarations that have equivalent 
parameter declarations \added{and associated constraints, if any 
(\ref{temp.constr.decl})}, declare the same function and therefore cannot be 
overloaded: ...
\exitnote
\end{quote}


%%
%% Declaration matching
%%
\rSec2[over.dcl]{Declaration matching}

Modify paragraph 1 to extend the notion of declaration matching to
also include a function's associated constrains. Note that the
example in the original text is omitted in this document.

\begin{quote}
Two function declarations of the same name refer to the same function if 
they are in the same scope and have equivalent parameter declarations 
(\ref{over.load}) \added{and equivalent associated constraints, if any 
(\ref{temp.constr.decl})}.
\end{quote}


%%
%% Overload resolution
%%
\setcounter{section}{2}
\rSec1[over.match]{Overload resolution}


%%
%% Viable functions
%%
\setcounter{subsection}{1}
\rSec2[over.match.viable]{Viable functions}

Update paragraph 1 to require the checking of a candidate's associated
constraints when determining if that candidate is viable.

\begin{quote}
\pnum
From the set of candidate functions constructed for a given context 
(\cxxref{over.match.funcs}), a set of viable functions is chosen, from which 
the best function will be selected by comparing argument conversion 
sequences and associated constraints for the best fit 
(\ref{over.match.best}).
%
The selection of viable functions considers \added{associated constraints, 
if any (\ref{temp.constr.decl}), and} relationships between arguments and function 
parameters other than the ranking of conversion sequences.
\end{quote}

Insert a new paragraph after paragraph 2; this introduces new a criterion for 
determining if a candidate is viable. Also, update the beginning of the 
subsequent paragraph to account for the insertion.

\begin{quote}
\setcounter{Paras}{2}
\pnum
\added{Second, for a function to be viable, if it has associated constraints, 
those constraints shall be satisfied (\ref{temp.constr.decl}).}

\pnum
\removed{Second}\added{Third}, for \tcode{F} to be a viable function\ldots
\end{quote}

%%
%% Best viable function
%%
\rSec2[over.match.best]{Best viable function}

Modify the last item in the list in paragraph 1 and extend it with a final
comparison based on the associated constraints of those functions. Note that 
the preceding (unmodified) items in the \Cpp Standard are elided in this
document.

\begin{quote}
\begin{itemize}
\item \ldots

\item \tcode{F1} and \tcode{F2} are function template specializations, and 
the function template for \tcode{F1} is more specialized than the template 
for \tcode{F2} according to the partial ordering rules described in
\ref{temp.func.order}\removed{.}\added{, or, if not that,}

\item
\added{\tcode{F1} and \tcode{F2} are non-template functions with the same 
parameter-type-lists, and  \tcode{F1} is more constrained than \tcode{F2} 
according to the partial ordering of constraints described in 
\ref{temp.constr.order}.}
\end{itemize}
\end{quote}


%%
%% Address of overloaded function
%%
\rSec1[over.over]{Address of overloaded function}

Modify paragraph 4 (paragraph 5 in this document) to incorporate 
constraints in the selection of an overloaded function when its address is 
taken.

% TODO: Rewrite this as an itemized list of steps.
\begin{quote}
\setcounter{Paras}{3}
\pnum
\added{Eliminate from the set of selected functions all those whose 
associated constraints are not satisfied (\ref{temp.constr.decl}).}
%
\removed{If more than one function is selected}
\added{If more than one function in the set remains}, 
any function template specializations in the set are eliminated if the 
set also contains a function that is not a function template
specialization\removed{, and}\added{.}
% 
\added{Any given non-template function \tcode{F0} is eliminated if the set 
contains a second non-template function that is more constrained than 
\tcode{F0} according to the partial ordering rules of \ref{temp.constr.order}.}
%
\added{Additionally,} any given function template specialization
\tcode{F1} is eliminated if the set contains a second function 
template specialization whose function template is more specialized than 
the function template of \tcode{F1} according to the partial ordering 
rules of \ref{temp.func.order}.
%
After such eliminations, if any, there shall remain exactly one 
selected function.
\end{quote}

Add the following example at the end of paragraph 5.

\begin{quote}
\begin{addedblock}
\enterexample
\begin{codeblock}
void f();                // \#1
void f() requires true ; // \#2
void g() requires false;
void g() requires false and true;

void (*pf)() = &f;         // selects \#2
void (*pg)() = &g;         // error: no matching function
\end{codeblock}
\exitexample
\end{addedblock}
\end{quote}
